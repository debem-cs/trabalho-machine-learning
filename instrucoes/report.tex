\documentclass{article}
\usepackage[utf8]{inputenc}
\usepackage[T1]{fontenc}
\usepackage[english]{babel}
\usepackage{graphicx}
\usepackage{hyperref}
\usepackage{geometry}
\usepackage{float}
\usepackage{listings}
\usepackage{xcolor}

\geometry{a4paper, margin=0.5in}

\title{Change Detection and Land Use Classification}
\author{}
\date{}

\begin{document}

\maketitle

\section{Overview}
Analysing changes in multi-image, multi-date remote sensing data helps us to discover and understand global conditions. This challenge uses satellite imagery-derived geographical features. The data has been processed using computer vision techniques and is ready for exploration using machine learning methods.

The aim of this challenge is to classify a given geographical area into six categories.

\section{Classes}
This challenge aims to classify a given geographical area into six classes:

\begin{itemize}
    \item \textbf{Demolition}: 0
    \item \textbf{Road}: 1
    \item \textbf{Residential}: 2
    \item \textbf{Commercial}: 3
    \item \textbf{Industrial}: 4
    \item \textbf{Mega Projects}: 5
\end{itemize}

\section{Data Description}
The training and test sets are provided in the \texttt{train.geojson} and \texttt{test.geojson} files, respectively.

\subsection{Features}
The geographical features are:
\begin{enumerate}
    \item An irregular polygon (\texttt{geometry}).
    \item Categorical values describing the status of the polygon on five different dates (e.g., under construction on day 0, completed on the following dates).
    \item Neighbourhood urban features (e.g., the polygon is in a dense urban and industrial region).
    \item Neighbourhood geographic features (e.g., the polygon is near a river and a hill).
\end{enumerate}

\subsection{Dataset Columns}
The columns available in the geojson files are:
\begin{itemize}
    \item \texttt{date0} to \texttt{date4}: Observation dates (DD-MM-YYYY).
    \item \texttt{change\_status\_date0} to \texttt{change\_status\_date4}: Status of polygon on each date.
    \item \texttt{urban\_type}: Comma-separated multiple values showing neighbourhood urban types.
    \item \texttt{geography\_type}: Comma-separated multiple values showing neighbourhood geographic types.
    \item \texttt{geometry}: Vector representation of geographic polygons.
    \item \texttt{change\_type}: Label to be classified (training set only).
\end{itemize}

For each date, scalar mean and standard deviation values of the colour image taken from a 50cm satellite are also provided:
\begin{itemize}
    \item Means: \texttt{img\_red\_mean\_date1} to \texttt{img\_red\_mean\_date5}, etc.
    \item Standard Deviations: \texttt{img\_red\_std\_date1} to \texttt{img\_red\_std\_date5}, etc.
\end{itemize}
\textbf{Important Note:} Please note that image statistics columns use suffixes from \texttt{date1} to \texttt{date5}, while date and status columns use \texttt{date0} to \texttt{date4}.

\section{Pipeline}
The proposed pipeline is the one introduced in the first lecture of the ML course:
\begin{enumerate}
    \item \textbf{Data Preprocessing}: You need to preprocess the data and convert it into the appropriate format.
    \item \textbf{Feature Engineering and Dimensionality Reduction}:
    \begin{itemize}
        \item Explore One-Hot Encoding for urban and geographic types.
        \item Create geometric features (area, perimeter) from the polygons.
        \item Calculate the number of days between two consecutive dates.
        \item Feature selection or dimensionality reduction could be beneficial.
    \end{itemize}
    \item \textbf{Learning Algorithm}:
    \begin{itemize}
        \item A simple baseline (k-NN) is provided in \texttt{skeleton\_code.py} ($\approx 40\%$ performance). The code uses only the \textbf{polygon area} as a feature.
        \item Test other classifiers: Logistic Regression, SVM, Decision Trees, Neural Networks, Ensemble Learning.
    \end{itemize}
    \item \textbf{Evaluation}: The evaluation metric is the \textbf{Mean F1-Score}.
\end{enumerate}

\section{Submission and Grading}
Form groups of 3-4 students.

\subsection{Deliverables}
\begin{enumerate}
    \item \textbf{Kaggle Submission} (40 points):
    \begin{itemize}
        \item Command to download data: \\ \texttt{kaggle competitions download -c 2-el-1730-machine-learning-project-2026}
        \item \textbf{Submission Step-by-Step:}
        \begin{enumerate}
            \item Run your trained model on the test data (\texttt{test.geojson}).
            \item Get the predicted class labels for each instance.
            \item Create a \texttt{sample\_submission.csv} file containing the header and two columns: \texttt{Id} and \texttt{change\_type}.
            \item Create an account on Kaggle (one for all team members) and create a new submission by uploading this file.
        \end{enumerate}
        \item Kaggle will automatically evaluate your predictions:
        \begin{itemize}
            \item Public Leaderboard (during competition): based on $\approx 30\%$ of test data.
            \item Private Leaderboard (final results): based on the remaining $70\%$.
        \end{itemize}
        \item Daily limit: 10 submissions.
    \end{itemize}
    
    \item \textbf{Report on Edunao} (50 points):
    \begin{itemize}
        \item PDF file only.
        \item Filename: \texttt{teamname\_student1name\_student2name\_student3name.pdf}.
        \item Must include full names and Kaggle team name.
        \item \textbf{Section 1: Feature Engineering}: Motivation, intuition, experiments, combinations of features tested.
        \item \textbf{Section 2: Model Tuning and Comparison}: Compare multiple classifiers, parameter tuning procedure, cross-validation, discussion of discarded models.
    \end{itemize}
    
    \item \textbf{Code on Edunao} (10 points - Global Grading):
    \begin{itemize}
        \item ZIP file: \texttt{name\_of\_your\_team.zip} (max 512MB).
        \item Reproducible code.
        \item The 10 points also cover respecting submission guidelines, organization, and clarity.
    \end{itemize}
\end{enumerate}

\begin{figure}[H]
    \centering
    \includegraphics[width=0.8\textwidth]{classes.png}
    \caption{Samples from 'test.geojson' showing different change types, status, dates, and neighbourhood labels.}
    \label{fig:classes}
\end{figure}

\section{References}
The dataset is part of the paper \textit{"QFabric: Multi-Task Change Detection Dataset"} (CVPR 2021W) by Sagar Verma et al.

\end{document}
